% Options for packages loaded elsewhere
\PassOptionsToPackage{unicode}{hyperref}
\PassOptionsToPackage{hyphens}{url}
%
\documentclass[
]{article}
\usepackage{amsmath,amssymb}
\usepackage{iftex}
\ifPDFTeX
  \usepackage[T1]{fontenc}
  \usepackage[utf8]{inputenc}
  \usepackage{textcomp} % provide euro and other symbols
\else % if luatex or xetex
  \usepackage{unicode-math} % this also loads fontspec
  \defaultfontfeatures{Scale=MatchLowercase}
  \defaultfontfeatures[\rmfamily]{Ligatures=TeX,Scale=1}
\fi
\usepackage{lmodern}
\ifPDFTeX\else
  % xetex/luatex font selection
\fi
% Use upquote if available, for straight quotes in verbatim environments
\IfFileExists{upquote.sty}{\usepackage{upquote}}{}
\IfFileExists{microtype.sty}{% use microtype if available
  \usepackage[]{microtype}
  \UseMicrotypeSet[protrusion]{basicmath} % disable protrusion for tt fonts
}{}
\makeatletter
\@ifundefined{KOMAClassName}{% if non-KOMA class
  \IfFileExists{parskip.sty}{%
    \usepackage{parskip}
  }{% else
    \setlength{\parindent}{0pt}
    \setlength{\parskip}{6pt plus 2pt minus 1pt}}
}{% if KOMA class
  \KOMAoptions{parskip=half}}
\makeatother
\usepackage{xcolor}
\usepackage[margin=1in]{geometry}
\usepackage{color}
\usepackage{fancyvrb}
\newcommand{\VerbBar}{|}
\newcommand{\VERB}{\Verb[commandchars=\\\{\}]}
\DefineVerbatimEnvironment{Highlighting}{Verbatim}{commandchars=\\\{\}}
% Add ',fontsize=\small' for more characters per line
\usepackage{framed}
\definecolor{shadecolor}{RGB}{248,248,248}
\newenvironment{Shaded}{\begin{snugshade}}{\end{snugshade}}
\newcommand{\AlertTok}[1]{\textcolor[rgb]{0.94,0.16,0.16}{#1}}
\newcommand{\AnnotationTok}[1]{\textcolor[rgb]{0.56,0.35,0.01}{\textbf{\textit{#1}}}}
\newcommand{\AttributeTok}[1]{\textcolor[rgb]{0.13,0.29,0.53}{#1}}
\newcommand{\BaseNTok}[1]{\textcolor[rgb]{0.00,0.00,0.81}{#1}}
\newcommand{\BuiltInTok}[1]{#1}
\newcommand{\CharTok}[1]{\textcolor[rgb]{0.31,0.60,0.02}{#1}}
\newcommand{\CommentTok}[1]{\textcolor[rgb]{0.56,0.35,0.01}{\textit{#1}}}
\newcommand{\CommentVarTok}[1]{\textcolor[rgb]{0.56,0.35,0.01}{\textbf{\textit{#1}}}}
\newcommand{\ConstantTok}[1]{\textcolor[rgb]{0.56,0.35,0.01}{#1}}
\newcommand{\ControlFlowTok}[1]{\textcolor[rgb]{0.13,0.29,0.53}{\textbf{#1}}}
\newcommand{\DataTypeTok}[1]{\textcolor[rgb]{0.13,0.29,0.53}{#1}}
\newcommand{\DecValTok}[1]{\textcolor[rgb]{0.00,0.00,0.81}{#1}}
\newcommand{\DocumentationTok}[1]{\textcolor[rgb]{0.56,0.35,0.01}{\textbf{\textit{#1}}}}
\newcommand{\ErrorTok}[1]{\textcolor[rgb]{0.64,0.00,0.00}{\textbf{#1}}}
\newcommand{\ExtensionTok}[1]{#1}
\newcommand{\FloatTok}[1]{\textcolor[rgb]{0.00,0.00,0.81}{#1}}
\newcommand{\FunctionTok}[1]{\textcolor[rgb]{0.13,0.29,0.53}{\textbf{#1}}}
\newcommand{\ImportTok}[1]{#1}
\newcommand{\InformationTok}[1]{\textcolor[rgb]{0.56,0.35,0.01}{\textbf{\textit{#1}}}}
\newcommand{\KeywordTok}[1]{\textcolor[rgb]{0.13,0.29,0.53}{\textbf{#1}}}
\newcommand{\NormalTok}[1]{#1}
\newcommand{\OperatorTok}[1]{\textcolor[rgb]{0.81,0.36,0.00}{\textbf{#1}}}
\newcommand{\OtherTok}[1]{\textcolor[rgb]{0.56,0.35,0.01}{#1}}
\newcommand{\PreprocessorTok}[1]{\textcolor[rgb]{0.56,0.35,0.01}{\textit{#1}}}
\newcommand{\RegionMarkerTok}[1]{#1}
\newcommand{\SpecialCharTok}[1]{\textcolor[rgb]{0.81,0.36,0.00}{\textbf{#1}}}
\newcommand{\SpecialStringTok}[1]{\textcolor[rgb]{0.31,0.60,0.02}{#1}}
\newcommand{\StringTok}[1]{\textcolor[rgb]{0.31,0.60,0.02}{#1}}
\newcommand{\VariableTok}[1]{\textcolor[rgb]{0.00,0.00,0.00}{#1}}
\newcommand{\VerbatimStringTok}[1]{\textcolor[rgb]{0.31,0.60,0.02}{#1}}
\newcommand{\WarningTok}[1]{\textcolor[rgb]{0.56,0.35,0.01}{\textbf{\textit{#1}}}}
\usepackage{graphicx}
\makeatletter
\def\maxwidth{\ifdim\Gin@nat@width>\linewidth\linewidth\else\Gin@nat@width\fi}
\def\maxheight{\ifdim\Gin@nat@height>\textheight\textheight\else\Gin@nat@height\fi}
\makeatother
% Scale images if necessary, so that they will not overflow the page
% margins by default, and it is still possible to overwrite the defaults
% using explicit options in \includegraphics[width, height, ...]{}
\setkeys{Gin}{width=\maxwidth,height=\maxheight,keepaspectratio}
% Set default figure placement to htbp
\makeatletter
\def\fps@figure{htbp}
\makeatother
\setlength{\emergencystretch}{3em} % prevent overfull lines
\providecommand{\tightlist}{%
  \setlength{\itemsep}{0pt}\setlength{\parskip}{0pt}}
\setcounter{secnumdepth}{-\maxdimen} % remove section numbering
\ifLuaTeX
  \usepackage{selnolig}  % disable illegal ligatures
\fi
\usepackage{bookmark}
\IfFileExists{xurl.sty}{\usepackage{xurl}}{} % add URL line breaks if available
\urlstyle{same}
\hypersetup{
  pdftitle={Assignment Part III},
  pdfauthor={Silpa Soni Nallacheruvu (19980824-5287) Hernan(20000526-4999)},
  hidelinks,
  pdfcreator={LaTeX via pandoc}}

\title{Assignment Part III}
\author{Silpa Soni Nallacheruvu (19980824-5287) Hernan(20000526-4999)}
\date{2024-10-21}

\begin{document}
\maketitle

set.seed(980824)

\section{Summary}\label{summary}

\section{Task 1}\label{task-1}

\subsection{Approach :}\label{approach}

To calculate the AIC, the formula

\(2*k - 2*l(\hat{\theta_{ml}})\)

was applied where k is the number of parameters in the theta. To
calculate the leave-one-out cross-validation, the formula

\(\frac{\sum_{i=1}^{n} {l_i(\theta_i)}}{n}\)
\(l(\hat{\theta_{ml}}) + \sum_{i=1}^{n} (({S_i(\theta_i)}^T)*(\hat{\theta_i} - \hat{\theta_{ml}}))\)

was applied where \(\hat{\theta_i}\) is \(\hat{\theta_{ml}}(X_{-i})\),
\((X_{-i})\) is one observation \((X_{i})\) is left out and n is the
total number of observations. These values were then compared with the
AIC from R in the provided summary.

\subsection{Code :}\label{code}

\begin{Shaded}
\begin{Highlighting}[]
\CommentTok{\# {-}{-}{-}{-} Task\_1 {-}{-}{-}{-}}

\CommentTok{\# Compute AIC = 2k {-} 2l(theta\_ml)}
\NormalTok{n }\OtherTok{\textless{}{-}} \FunctionTok{nrow}\NormalTok{(X)}
\NormalTok{theta0 }\OtherTok{=} \FunctionTok{rep}\NormalTok{(}\DecValTok{0}\NormalTok{, }\FunctionTok{ncol}\NormalTok{(X))}
\NormalTok{k }\OtherTok{\textless{}{-}} \FunctionTok{length}\NormalTok{(theta0)}
\NormalTok{theta\_estimate }\OtherTok{\textless{}{-}} \FunctionTok{NR}\NormalTok{(theta0, }\DecValTok{3}\NormalTok{, y, X) }
\NormalTok{log\_likelihood }\OtherTok{\textless{}{-}} \FunctionTok{l}\NormalTok{(theta\_estimate, y, X)}
\NormalTok{aic\_computed }\OtherTok{\textless{}{-}} \DecValTok{2}\SpecialCharTok{*}\NormalTok{k }\SpecialCharTok{{-}} \DecValTok{2}\SpecialCharTok{*}\NormalTok{log\_likelihood}

\CommentTok{\#AIC output from R summary }
\NormalTok{r\_summary\_aic }\OtherTok{\textless{}{-}} \FunctionTok{summary}\NormalTok{(modell)}\SpecialCharTok{$}\NormalTok{aic}

\CommentTok{\# Compute n*k\_cv = l(theta\_ml) + sum((S\_i(theta\_i))T*(theta\_i {-} theta\_ml))}
\CommentTok{\# Here, theta\_i = theta\_ml(X\_{-}i)}
\NormalTok{nk\_cv }\OtherTok{\textless{}{-}}\NormalTok{ log\_likelihood}
\ControlFlowTok{for}\NormalTok{(i }\ControlFlowTok{in} \DecValTok{1}\SpecialCharTok{:}\NormalTok{n) \{}
\NormalTok{  X\_minus\_i }\OtherTok{\textless{}{-}}\NormalTok{ X[}\SpecialCharTok{{-}}\NormalTok{i, , drop}\OtherTok{=}\ConstantTok{FALSE}\NormalTok{]}
\NormalTok{  y\_minus\_i }\OtherTok{\textless{}{-}}\NormalTok{ y[}\SpecialCharTok{{-}}\NormalTok{i, , drop}\OtherTok{=}\ConstantTok{FALSE}\NormalTok{]}
\NormalTok{  theta\_i }\OtherTok{\textless{}{-}} \FunctionTok{NR}\NormalTok{(theta0, }\DecValTok{3}\NormalTok{, y\_minus\_i, X\_minus\_i)}
\NormalTok{  score }\OtherTok{\textless{}{-}} \FunctionTok{S}\NormalTok{(theta\_i, y\_minus\_i, X\_minus\_i)}
\NormalTok{  subset\_sum }\OtherTok{\textless{}{-}} \FunctionTok{t}\NormalTok{(score) }\SpecialCharTok{\%*\%}\NormalTok{ (theta\_i }\SpecialCharTok{{-}}\NormalTok{ theta\_estimate)}
\NormalTok{  nk\_cv }\OtherTok{\textless{}{-}}\NormalTok{ nk\_cv }\SpecialCharTok{+}\NormalTok{ subset\_sum}
\NormalTok{\}}

\CommentTok{\# Creating a comparison data frame }
\NormalTok{comparison\_aic\_values }\OtherTok{\textless{}{-}} \FunctionTok{data.frame}\NormalTok{(}
  \StringTok{"AIC\_R\_model"} \OtherTok{=}\NormalTok{ r\_summary\_aic,}
  \StringTok{"AIC\_computed"} \OtherTok{=}\NormalTok{ aic\_computed,}
  \StringTok{"2*nK\_CV\_computed"} \OtherTok{=} \DecValTok{2}\SpecialCharTok{*}\NormalTok{nk\_cv}
\NormalTok{)}
\end{Highlighting}
\end{Shaded}

\subsection{Output :}\label{output}

\begin{Shaded}
\begin{Highlighting}[]
\NormalTok{comparison\_aic\_values}
\end{Highlighting}
\end{Shaded}

\begin{verbatim}
##   AIC_R_model AIC_computed X2.nK_CV_computed
## 1    1302.397     1302.397         -1294.397
\end{verbatim}

\subsection{Observation :}\label{observation}

The computed AIC value and the AIC value from the R summary coincide in
value. The computed nK\_CV is in the same magnitude as \(AIC/(-2)\) as
expected.

\section{Task 2}\label{task-2}

\subsection{Approach :}\label{approach-1}

The a posteriori density combines the a priori and the likelihood of the
data. For the logistic regression, we have y data, X and the parameter
vector \(_theta\). so the a posteriori density is proportional to:
\$P(\theta\textbar y,X) \propto P(y\textbar X,\theta)P(\theta) \$ where
\(P(\theta)\) is the a prior density and \(P(y1X,\theta)\) is. the
likelihood from the logistic regression model.

for a binary outcome, the likelihood for the logistic regression is
given by the following equation:

\(P(y_{i}|X_{i},\theta) =  \frac{1}{(1+\exp(-X_{i}\theta)}\)

The a priori is Gaussian: \(\theta N(0, 100I)\) so the a prior density
is the following:
\(P(\theta) \propto exp(-\frac{1}{2}\theta^T(100I)^{-1}\theta)\)

\subsection{Code :}\label{code-1}

\begin{Shaded}
\begin{Highlighting}[]
\CommentTok{\# {-}{-}{-}{-} Task\_2 {-}{-}{-}{-}}
\NormalTok{post }\OtherTok{\textless{}{-}} \ControlFlowTok{function}\NormalTok{(theta, y, X) \{}
\NormalTok{  eta }\OtherTok{\textless{}{-}}\NormalTok{ X }\SpecialCharTok{\%*\%}\NormalTok{ theta }\CommentTok{\# Logistic regression likelihood}
\NormalTok{  likelihood }\OtherTok{\textless{}{-}} \FunctionTok{prod}\NormalTok{(}\FunctionTok{plogis}\NormalTok{(eta)}\SpecialCharTok{\^{}}\NormalTok{y }\SpecialCharTok{*}\NormalTok{ (}\DecValTok{1} \SpecialCharTok{{-}} \FunctionTok{plogis}\NormalTok{(eta))}\SpecialCharTok{\^{}}\NormalTok{(}\DecValTok{1} \SpecialCharTok{{-}}\NormalTok{ y))}
\NormalTok{  prior }\OtherTok{\textless{}{-}} \FunctionTok{exp}\NormalTok{(}\SpecialCharTok{{-}}\FloatTok{0.5} \SpecialCharTok{*} \FunctionTok{sum}\NormalTok{(theta}\SpecialCharTok{\^{}}\DecValTok{2} \SpecialCharTok{/} \DecValTok{100}\NormalTok{)) }\CommentTok{\# a priori with theta \textasciitilde{} N(0, 100 * I)}
\NormalTok{  posterior }\OtherTok{\textless{}{-}}\NormalTok{ likelihood }\SpecialCharTok{*}\NormalTok{ prior }\CommentTok{\# posteriori is proportional to a priori * likelihood}
  \FunctionTok{return}\NormalTok{(posterior)}
\NormalTok{\}}
\end{Highlighting}
\end{Shaded}

\subsection{Output :}\label{output-1}

\begin{Shaded}
\begin{Highlighting}[]
\NormalTok{Xtest }\OtherTok{\textless{}{-}} \FunctionTok{cbind}\NormalTok{(}\DecValTok{1}\NormalTok{, }\DecValTok{18}\SpecialCharTok{:}\DecValTok{25}\NormalTok{, }\FunctionTok{rep}\NormalTok{(}\FunctionTok{c}\NormalTok{(}\DecValTok{0}\NormalTok{, }\DecValTok{1}\NormalTok{), }\DecValTok{4}\NormalTok{), }\FunctionTok{rep}\NormalTok{(}\FunctionTok{c}\NormalTok{(}\DecValTok{1}\NormalTok{, }\DecValTok{1}\NormalTok{, }\DecValTok{0}\NormalTok{, }\DecValTok{0}\NormalTok{), }\DecValTok{2}\NormalTok{))}
\NormalTok{ytest }\OtherTok{\textless{}{-}} \FunctionTok{c}\NormalTok{(}\FunctionTok{rep}\NormalTok{(}\ConstantTok{TRUE}\NormalTok{, }\DecValTok{4}\NormalTok{), }\FunctionTok{rep}\NormalTok{(}\ConstantTok{FALSE}\NormalTok{, }\DecValTok{4}\NormalTok{))}
\NormalTok{testing}\OtherTok{\textless{}{-}}\FunctionTok{post}\NormalTok{(}\FunctionTok{c}\NormalTok{(}\DecValTok{260}\NormalTok{, }\SpecialCharTok{{-}}\DecValTok{10}\NormalTok{, }\DecValTok{10}\NormalTok{, }\SpecialCharTok{{-}}\DecValTok{20}\NormalTok{), ytest, Xtest) }\SpecialCharTok{/} \FunctionTok{post}\NormalTok{(}\FunctionTok{c}\NormalTok{(}\DecValTok{270}\NormalTok{, }\SpecialCharTok{{-}}\DecValTok{15}\NormalTok{, }\DecValTok{15}\NormalTok{, }\SpecialCharTok{{-}}\DecValTok{25}\NormalTok{), ytest , Xtest)}
\NormalTok{testing}
\end{Highlighting}
\end{Shaded}

\begin{verbatim}
## [1] 3.707556e+25
\end{verbatim}

\subsection{Observation :}\label{observation-1}

Given the results obtained by the test the function works correctly. \#
Task 3

\subsection{Approach :}\label{approach-2}

\subsection{Code :}\label{code-2}

\begin{Shaded}
\begin{Highlighting}[]
\CommentTok{\# {-}{-}{-}{-} Task\_3 {-}{-}{-}{-}}
\end{Highlighting}
\end{Shaded}

\subsection{Output :}\label{output-2}

\subsection{Observation :}\label{observation-2}

\end{document}
